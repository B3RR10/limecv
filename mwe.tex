% !TEX TS-program = xelatex
% !TEX encoding = UTF-8 Unicode

% This is a MWE of the `limecv' class that showcases its features and sane packages to load. This MWE assumes that the XeLaTeX compiler is used.


\documentclass[a4paper]{limecv}

% Defaults used in template design.
\usepackage[margin=1cm,noheadfoot]{geometry}

% For cover letter.
\usepackage{lipsum}

\begin{document}

% Design of side bar.
\begin{cvSideBar}
  \cvID{John}{Doe}{picture}{position}

  \begin{cvProfile}
    Lorem ipsum dolor sit amet, consectetur adipiscing elit. Phasellus ullamcorper euismod lorem nec eleifend. Suspendisse ac varius quam. Etiam laoreet nunc orci, vestibulum imperdiet enim elementum at. Duis dictum metus sapien, eu blandit quam malesuada et. Ut viverra maximus eros.
  \end{cvProfile}
  
  \begin{cvContact}
    \cvContactAddress{Some Street 78\\B-9000 Ghent\\Belgium}
    \cvContactEmail{mailto:john@doe.tld}{john@doe.tld}
    \cvContactPhone{+1 781 555 1212}
    \cvContactWebsite{https://doe.tld}{doe.tld}
    \cvContactGithub{https://github.com/johndoe}{johndoe}
    \cvContactLinkedin{https://www.linkedin.com/in/johndoe/}{johndoe}
    \cvContactTwitter{https://twitter.com/johndoe}{@johndoe}
    \cvContactKeybase{https://keybase.io/johndoe}{\texttt{AAAA 5555 BBBB FFFF}}
  \end{cvContact}
  
  \begin{cvLanguages}
    \cvLanguage{English (native)}{5}
    \cvLanguage{German (B2)}{3}
    \cvLanguage{Spanish}{3}
  \end{cvLanguages}

\end{cvSideBar}%
\hfill%
%
% Design of main section.
%
\begin{cvMainContent}
  % education section
  \begin{cvEducation}
    \cvItem{Evening class: Chinese\\
    Some School, City. September 2015 -- June 2016\\
    Achieved A2 language skill in Chinese (Mandarin).}
    \cvItem{Bachelor of Science in Biochemistry and Biotechnology\\
    University, City. September 2009 -- June 2012\\
    General training in the basic sciences and the molecular life science.}
    \cvItem{Master of Science in Biochemistry and Biotechnology\\
    University, City. September 2012 -- June 2015\\
    Acquisition of insight into and knowledge of possibilities for application in the area of biochemistry and biotechnology, specific with applications in biomedical application and due problem-solving reasoning skills.}
  \end{cvEducation}
  
  % experience section
  \begin{cvExperience}
    \cvItem{Student Job\\
            \textsc{\selectfont Company X}, Location X. Summer 2010\\
            Integer tincidunt dapibus consectetur. Nullam tristique aliquam luctus. Sed ut ante velit. Nulla pharetra maximus lacus at elementum. Suspendisse sodales consectetur metus, sit amet ultricies ipsum ultrices ut.};
    \cvItem{Internship\\
            \textsc{Company Y}, Location Y. June 2012 -- August 2012\\
            Lorem ipsum dolor sit amet, consectetur adipiscing elit. Morbi dictum cursus sapien, id eleifend mi pellentesque id. Etiam lobortis eu odio a sodales. Phasellus ut dolor feugiat, lacinia lectus in, blandit metus. Fusce lacinia dolor et metus gravida pulvinar sit amet et ex.};
    \cvItem{Internship\\
            \textsc{Company Z}, Location Z. August 2014 -- September 2014\\
            Lorem ipsum dolor sit amet, consectetur adipiscing elit. Morbi dictum cursus sapien, id eleifend mi pellentesque id. Etiam lobortis eu odio a sodales. Phasellus ut dolor feugiat, lacinia lectus in, blandit metus.  Fusce lacinia dolor et metus gravida pulvinar sit amet et ex. Suspendisse vestibulum, leo malesuada molestie maximus, sem risus ornare elit, vitae sodales felis elit in ipsum.};
  \end{cvExperience}
  
  \begin{cvSkills*}
    \cvSkillTwo*{5}{MATLAB}{5}{\ LaTeX}
    \cvSkillTwo*{4}{Python}{4}{VHDL}
    \cvSkillTwo*{4}{Microsoft Office}{4}{macOS}
    \cvSkillTwo*{3}{C, C++}{1}{Javascript}
    \cvSkillTwo*{3}{HTML5/CSS}{3}{Bash}
  \end{cvSkills*}

  \begin{cvReferences}
    \cvAddReference[
      name=Jane Smith,
      company=Company ABC Co.\ Ltd.,
      job=Job title,
      address line 1=Street lane 2,
      address line 2=B-1150 Brussels,
      mobile phone=+1 781 555 1212]%
  \end{cvReferences}

\end{cvMainContent}

% Force a new page.
\clearpage
\newpage

% Main content section on left side.
\begin{cvMainContent*}

\end{cvMainContent*}%
\hfill%
%
% Side bar on right side
%
\begin{cvSideBar*}

  \begin{cvInterests}
    \cvInterest{\faTrain}{model trains}
    \cvInterest{\faMicrochip}{electronics}
    \cvInterest{\faFlask}{(applied) sciences}
    \cvInterest{\faSuitcase}{travelling}
    \cvInterest{\faFilm}{film}
    \cvInterest{\faCamera}{photography}
  \end{cvInterests}

  \begin{cvProjects}
  \end{cvProjects}

\end{cvSideBar*}

\clearpage

\newlength\coverletterheight
\setlength\coverletterheight{\sidewidth}
\newlength\coverletterwidth
\setlength\coverletterwidth{\mainwidth+3\margin}

\begin{tikzpicture}[remember picture,overlay]
  \begin{scope}[on background layer]
  \draw (current page.center) ++(-\paperwidth/2,\paperheight/2) node (h5) {};
  \draw (h5) ++(\paperwidth,-\coverletterheight) node (h6) {};
  \draw[fill=cvGreenLight,cvGreenLight] (h5) rectangle (h6);
    \end{scope}
  \draw (current page.north east) ++(-0.5\paperwidth+0.5\coverletterwidth,-\coverletterheight/2) node (h7) {};
  \matrix[anchor=east,row sep=10pt] at (h7) {%
  \node (cv cover letter name){\fontsize{50}{60}\selectfont
  \pgfkeysvalueof{/cv/info/first name}
  \pgfkeysvalueof{/cv/info/last name}}; \\
  \node[align=right,cvAccent]{\pgfkeysvalueof{/cv/info/position}}; \\
};
  \begin{scope}[on background layer]
    \draw[line width=3pt,cvGreen] 
      (cv cover letter name.south west) to (cv cover letter name.south east);
  \end{scope}
\end{tikzpicture}

\vspace{\dimexpr\coverletterheight\relax}

\begin{center}
\begin{minipage}{\coverletterwidth}
\today

\vspace{\baselineskip}


\NewDocumentCommand{\cvBeneficiary}{mmmmmm}{%
\begin{tabular}{@{}l}
\MakeUppercase{#1 #2} \\
#3 \\
#4 \\
#5 \\
#6 \\
\end{tabular}%
}
\cvBeneficiary{Jane}{Smith}{Position}{Company}{Address line 1}{Address line 2}

\vspace{\margin}

Dear Miss.\ Smith

\vspace{\baselineskip}

\lipsum[1-3]

\vspace{\margin}

\pgfkeysvalueof{/cv/info/first name} \pgfkeysvalueof{/cv/info/last name}

\end{minipage}
\end{center}

\end{document}